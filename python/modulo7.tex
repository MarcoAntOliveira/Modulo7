
\documentclass{article}
\usepackage[legalpaper, left=1 cm, right=1cm, top=0.5cm, bottom=0.5cm] {geometry}
\date{} % Remove a exibição da data
\usepackage{xcolor}
\usepackage{listings}
\usepackage{graphicx}
\usepackage{hyperref} % Para criar links
\usepackage[utf8]{inputenc}
\usepackage[T1]{fontenc}

\lstdefinestyle{pythonStyle}{
    language=Python,
    basicstyle=\ttfamily\small,
    keywordstyle=\color{blue},
    stringstyle=\color{red},
    commentstyle=\color{blue}\itshape,
    numbers=left,
    numberstyle=\tiny\color{blue},
    breaklines=true,
    showstringspaces=false,
}

% Define o estilo para HTML
\lstdefinestyle{htmlStyle}{
    language=HTML,
    basicstyle=\ttfamily\small,
    keywordstyle=\color{blue},
    stringstyle=\color{orange},
    commentstyle=\color{gray}\itshape,
    numbers=left,
    numberstyle=\tiny\color{red},
    breaklines=true,
    showstringspaces=false,
}

\lstdefinestyle{cssStyle}{
    language=HTML,
    basicstyle=\ttfamily\small,
    keywordstyle=\color{blue},
    stringstyle=\color{green!50!black},
    commentstyle=\color{gray}\itshape,
    numbers=left,
    numberstyle=\tiny\color{gray},
    breaklines=true,
    showstringspaces=false,
}


\title{Modulo 7}
\begin{document}
\maketitle
\tableofcontents
\section{introduction}
\begin{lstlisting}[style=pythonStyle, caption={comandos django}]
    django-admin  startproject project .# inicia o projeto django com o manage.py na raiz
    python manage.py runserver #sobe o servidor
\end{lstlisting}
\textbf{http codes}
\textcolor{red}{https://developer.mozilla.org/pt-BR/docs/Web/HTTP/Status}

\section{
    Primeira URL e function based view + HttpRequest e HttpResponse3
    }

\begin{lstlisting}[style=pythonStyle, caption={urls.py}]
    def home (request):
        print("home")
        return HttpResponse("HOME")

    def my_view(request):
        return HttpResponse("hello world")


    urlpatterns = [
        path('admin/', admin.site.urls),
        path('blog/', my_view),
        path('', home),
    ]
\end{lstlisting}

\section{Movendo as functions base views para os novos Apps no Django}
\textbf{comnandos}
\begin{lstlisting}[style=pythonStyle, caption={comando django}] 
django-admin startapp <nome_app> \end{lstlisting}

Aqui ocorre o aninhamento das urls que estarão nos apps
\begin{lstlisting}[style=pythonStyle, caption={project/urls.py}]
    from django.contrib import admin
    from django.urls import include ,path
    urlpatterns = [
        path('', include('home.urls')),
        path('admin/', admin.site.urls),
        path('blog/', include('blog.urls')),
    ]
\end{lstlisting}

Aqui são as responses de cada urls
\begin{lstlisting}[style=pythonStyle, caption={blog/views.py}]
    #from django.shortcuts import render
    from django.http import HttpResponse

    # Create your views here.
    def blog(request):
        return HttpResponse("BLOG")

    def example(request):
        return HttpResponse("BLOG/example")
\end{lstlisting}

\begin{lstlisting}[style=pythonStyle, caption={blog/urls.py}]
    from django.urls import path
    from blog.views import blog, example

    urlpatterns = [
        path('', blog),
        path('example/', example)
    ]
\end{lstlisting}
E a mesma coisa acontece para o home

\begin{lstlisting}[style=pythonStyle, caption={home/views.py}]
    from django.http import HttpResponse
    #from django.shortcuts import render

    # Create your views here.
    def home (request):#function base view
        print("home")
        return HttpResponse("HOME")

\end{lstlisting}

\begin{lstlisting}[style=pythonStyle, caption={home/urls.py}]
    from django.urls import path
    from home.views import home

    urlpatterns = [
        path('', home),
    ]
\end{lstlisting}

\section{Renderizando HTML, render, templates, INSTALLEDAPPS e TemplateDoesNotExist}

\begin{lstlisting}[style=pythonStyle, caption={project/settings.py}]
    INSTALLED_APPS = [
    'django.contrib.admin',
    'django.contrib.auth',
    'django.contrib.contenttypes',
    'django.contrib.sessions',
    'django.contrib.messages',
    'django.contrib.staticfiles',
    'home',
    'blog',
    ]#adicionando configuracoes dos apps
\end{lstlisting}
\textbf{um pasta com o nome do app, dentro da pasta templates, garante segurança na importação}

\begin{lstlisting}[style=htmlStyle, caption={blog/templates/blog/blog.html}]
    <!DOCTYPE html>
    <html lang="en">
    <head>
        <meta charset="UTF-8">
        <meta name="viewport" content="width=device-width, initial-scale=1.0">
        <title>Document</title>
    </head>
    <body>
        <b> BLOG</b>
    </html>
\end{lstlisting}


\begin{lstlisting}[style= htmlStyle, caption={blog/templates/blog/example.html}]
    <!DOCTYPE html>
    <html lang="en">
    <head>
        <meta charset="UTF-8">
        <meta name="viewport" content="width=device-width, initial-scale=1.0">
        <title>Document</title>
    </head>
    <body>
        <strong>Example</strong>
    </html>

\end{lstlisting}

\begin{lstlisting}[style=pythonStyle, caption={blog/views.py}]
    from django.shortcuts import render

    # Create your views here.
    def blog(request):
        return render(request,
                    'blog.html')

    def example(request):
        return render(request, 
                    'example.html')

\end{lstlisting}


\begin{lstlisting}[style=htmlStyle, caption={home/templates/home/home.html}]
    <!DOCTYPE html>
    <html lang="en">
    <head>
        <meta charset="UTF-8">
        <meta name="viewport" content="width=device-width, initial-scale=1.0">
        <title>Document</title>
    </head>
    <body>
        <strong>HOME</strong>
    </html>
\end{lstlisting}


\begin{lstlisting}[style=pythonStyle, caption={home/views.py}]
    from django.shortcuts import render

    # Create your views here.
    def home (request):#function base view
        print("home")
        return render(
            request,
            'home.html'
        )

\end{lstlisting}

\section{Configurando templates globais com DIRS + extends para herança de templates}


\begin{lstlisting}[style=pythonStyle, caption={project/settings.py}]
    # aqui estao as configuracoes dos templates usados no app
    TEMPLATES = [
{
    'BACKEND': 'django.template.backends.django.DjangoTemplates',
    'DIRS': [
        \textcolor{blue}{BASE_DIR / 'base'}
        ],
    'APP_DIRS': True,
    'OPTIONS': {
        'context_processors': [
            'django.template.context_processors.debug',
            'django.template.context_processors.request',
            'django.contrib.auth.context_processors.auth',
            'django.contrib.messages.context_processors.messages',
        ],
    },
},
]
\end{lstlisting}



\begin{lstlisting}[style=htmlStyle, caption={base/global/base.html}]
    <!DOCTYPE html>
    <html lang="en">
    <head>
        <meta charset="UTF-8">
        <meta name="viewport" content="width=device-width, initial-scale=1.0">
        <title>Document</title>
    </head>
    <body>
       \textcolor{blue}{<h1>  BASE </h1>} 
    </body>
    </html>
\end{lstlisting}

\begin{lstlisting}[style=htmlStyle, caption={home/views.py}]
    from django.shortcuts import render
    # Create your views here.
    def home (request):#function base view
        print("home")
        return render(
            request,
            'home/home.html'
        )

\end{lstlisting}
extends significa extender para o html da base

\begin{lstlisting}[style=htmlStyle, caption={home/templates/home.html}]
    
     MUDAR O texto

\end{lstlisting}

\section{configurando templates globais}

\begin{lstlisting}[style=htmlStyle, caption={base/global/base.html}]
    <!DOCTYPE html>
    <html lang="en">
    <head>
        <meta charset="UTF-8">
        <meta name="viewport" content="width=device-width, initial-scale=1.0">
        <title>Document</title>
    </head>
    <body>
        <h1>  BASE </h1>
    </body>
    </html>
\end{lstlisting}

\begin{lstlisting}[style=htmlStyle, caption={home/templates/home.html}]
    
     MUDAR O texto
\end{lstlisting}


\begin{lstlisting}[style=htmlStyle, caption={blog/templates/blog.html}]
    
     bem vindo ao blog 
\end{lstlisting}


\begin{lstlisting}[style=htmlStyle, caption={blog/templates/example.html}]
    
     Example 
\end{lstlisting}
\section{Arquivos parciais e includes para separar trechos dos templates(partials)}
\textbf{}
\begin{lstlisting}[style=htmlStyle, caption={base/global/partials/head.html}]
        <!DOCTYPE html>
    <html lang="en">
    <head>
        <meta charset="UTF-8">
        <meta name="viewport" content="width=device-width, initial-scale=1.0">
        <title>Document</title>
    </head>
\end{lstlisting}
\begin{lstlisting}[style=htmlStyle, caption={base/global/partials/paragrafo.html}]
    <p> um texto qualquer </p>
\end{lstlisting}


\begin{lstlisting}[style=htmlStyle, caption={base/global/base.html}]
    
    <h1>  BASE </h1>
    
    
    
    
</body>
</html>
\end{lstlisting}

\section{Arquivos estaticos (static files), STATICURL, STATICFILESDIRS,  load static}
configurando o settings do project para uma nova pasta static em base
\begin{lstlisting}[style=pythonStyle, caption={project/settings.py}]
    # Static files (CSS, JavaScript, Images)
    # https://docs.djangoproject.com/en/5.0/howto/static-files/

    STATIC_URL = 'static/'
    STATICFILES_DIRS = [
        BASE_DIR / 'base'/ 'static'
]
\end{lstlisting}

\begin{lstlisting}[style=cssStyle, caption={home/css/blue.css}]
    body{
        background: blue;
    }
\end{lstlisting}

\begin{lstlisting}[style=cssStyle, caption={global/css/red.css}]
    body{
        background: red;
    }
\end{lstlisting}

\begin{lstlisting}[style=htmlStyle, caption={base/global/partials/head.html}]
    <!DOCTYPE html>
<html lang="en">
<head>
    <meta charset="UTF-8">
    <meta name="viewport" content="width=device-width, initial-scale=1.0">
    <title>Document</title>
    <link rel = "stylesheet" href = "">
    <link rel = "stylesheet" href = "">
</head>
\end{lstlisting}

\section{Usando context para enviar dados para dentro do views}
\begin{lstlisting}[style=htmlStyle, caption={home/templates/home/home.html}]

 
{{ text }}

   
\end{lstlisting}

\begin{lstlisting}[style=pythonStyle, caption={home/views.py}]
    from django.shortcuts import render

    # Create your views here.
    def home (request):#function base view
        print("home")
        context = {
                'text' : 'estamos aqui'
            }
    
        return render(
            request,
            'home/home.html',
            context
        )
    
\end{lstlisting}

\begin{lstlisting}[style=htmlStyle, caption={blog/templates/blog/example.html}]
    
     {{ text }} 
\end{lstlisting}

\begin{lstlisting}[style=htmlStyle, caption={blog/templates/blog/example.html}]

 {{ text }}   
\end{lstlisting}

\begin{lstlisting}[style=pythonStyle, caption={blog/views.py}]
    from django.shortcuts import render

# Create your views here.
def blog(request):
    context = {
        'text' : 'ola aqui do blog'
    }
    return render(request,
                  'blog/blog.html', 
                  context
                  )

def example(request):
    context = {
        'text' : 'Example'
    }
    return render(request, 
                  'blog/example.html', 
                  context
                  )

\end{lstlisting}
\section{trabalhando com urls dinamicas}
\begin{lstlisting}[style=htmlStyle, caption={base/global/partials/menu.html}]
    <nav> 
    <ul>
        <li>
            <a href= "">Home</a>
        </li>   
        
        <li>
            <a href= "">Blog</a>
        </li>   

        <li>
            <a href= "">Example</a>
        </li>   
    </ul>    
</nav>
   
\end{lstlisting}



Adicionando namespace a url
\begin{lstlisting}[style=pythonStyle, caption={blog/urls.py}]
from django.urls import path
from blog.views import blog, example

app_name = 'blog'
urlpatterns = [
    path('', blog, name='home'),
    path('example/', example, name='example')
]
\end{lstlisting}

\begin{lstlisting}[style=pythonStyle, caption={home/urls.py}]
from django.urls import path
from home.views import home

app_name = 'home'
urlpatterns = [
    path('', home, name= 'index'),
]
\end{lstlisting}

\section{Movendo todos os arquivos de css para global}
\begin{lstlisting}[style=htmlStyle, caption={base/global/partials/head.html}]
    <!DOCTYPE html>
<html lang="en">
<head>
    <meta charset="UTF-8">
    <meta name="viewport" content="width=device-width, initial-scale=1.0">
    <title>{{title}} Site do Marcos </title>
    <link rel="stylesheet" href="">
</head>
\end{lstlisting}   

\begin{lstlisting}[style=cssStyle, caption={base/static/global/css/style.css}]
    *{
        margin: 0;
        padding: 0;
        box-sizing: border-box; 
    }    

\end{lstlisting}
\section{Criando o partial postblock.html e usando include}
\begin{lstlisting}[style=cssStyle, caption={base/static/global/css/style.css}]
    /* Reset */
*,
*:after,
*:before {
  margin: 0;
  padding: 0;
  box-sizing: border-box;
}

html {
  font-size: 62.5%;
}

body {
  font-size: 1.6rem;
  background: #f1f1f1;
  font-family: system-ui, -apple-system, BlinkMacSystemFont, 'Segoe UI', Roboto,
    Oxygen, Ubuntu, Cantarell, 'Open Sans', 'Helvetica Neue', sans-serif;
}
\end{lstlisting}

\begin{lstlisting}[style=htmlStyle, caption={base/global/base.html}]



        <h1>  BASE </h1>
        
        

        <main class="posts">
            
            
            
        

        </main>
        
        
        
    </body>
</html>
   
\end{lstlisting}

\begin{lstlisting}[style=htmlStyle, caption={base/global/partials/postblock.html}]
    <article>
    <header>
        <h2 class="post__title">
            Lorem ipsum dolor sit amet consectetur adipisicing elit. Natus placeat blanditiis ipsam quas est, provident, exercitationem illo inventore molestias iure beatae soluta aliquid iusto facere corporis, quaerat aspernatur debitis laudantium?
        </h2>
    </header>
    <div class="post__body">
        Lorem ipsum dolor sit amet consectetur adipisicing elit. Eum laboriosam beatae veritatis dolorem natus voluptatibus fugiat sequi eaque exercitationem dolor nobis assumenda facere, praesentium aspernatur id odit aliquam ipsa nisi.
    </div>
</article>

   
\end{lstlisting}

\section{usando block para criar blocos de posts e home }
\begin{lstlisting}[style=htmlStyle, caption={base/global/base.html}]
    
    
    
            <h1>  BASE </h1>
            
            
    
            <main class="content">
                 
                
                
            </main>
            
            
            
        </body>
    </html>
\end{lstlisting}

\begin{lstlisting}[style=htmlStyle, caption={blog/templates/blog/blog.html}]
    
     {{text}} 
    
    
    
    
    
    
    <h1> Blog </h1>
    
    
\end{lstlisting}

\section{Entendendo seu HTML final + adiconando css aos posts}
\textbf{.content altera os content da pagina html\\}
\textbf{.post altera diretamente os posts\\}
\textbf{@media altera o formato das caixas de posts}
\begin{lstlisting}[style=cssStyle, caption={base/static/global/css/style.css}]
    
.content {
    display: grid;
    gap: 1.5rem;
    padding: 1.5rem;
  }
  
  .post{
    background : #fff;
    padding: 1.5rem;
    box-shadow: 5px 2px 5px rgba(0, 0, 0, 0.9)
  }
  
  @media (min-width: 600px){
    .content{
      grid-template-columns: repeat(auto-fill, minmax(32rem, 1fr));
    }
  }
  
\end{lstlisting}

\section{Criando os dados de posts (data.py) e usando loop for no template}
\textbf{O codigo do data.py foi criado apartir de dados de uma api que usa json.}\\
\textbf{O codigo foi modificado pra que apartir dos dados do data.py, gere na pagina blog, os diversos posts.}
\begin{lstlisting}[style=htmlStyle, caption={base/global/partials/postblock.htmll}]
    <article class="post">
    <header>
      <h2 class="post__title">Lorem ipsum dolor sit amet.{{post.title}}</h2>
    </header>
    <div class="post__body">{{post.body}}</div>
  </article>
\end{lstlisting}

\begin{lstlisting}[style=htmlStyle, caption={blog/templates/blog/blog.html}]


 {{ text }} 





 
\end{lstlisting}

\begin{lstlisting}[style=pythonStyle, caption={blog/views.py}]
    def blog(request):
    context = {
        'text' : 'ola aqui do blog',
        'posts' : data.posts
    }
    return render(request,
                  'blog/blog.html', 
                  context
                  )

\end{lstlisting}

\section{Usando if, elif, e else dentro do template}
\textbf{No exemplo passado em aula, o if é usado para checar a existencia da da variavel text, se não existir ela nao aparece na pagina html}

\begin{lstlisting}[style=pythonStyle, caption={blog/views.py}]
    from django.shortcuts import render
from . import data
# Create your views here.
def blog(request):
    context = {
        'text' : 'ola aqui do blog',
        'posts' : data.posts
    }
    return render(request,
                  'blog/blog.html', 
                  context
                  )

def example(request):
    context = {
        'text' : 'Example'
    }
    return render(request, 
                  'blog/example.html', 
                  context
                  )

\end{lstlisting}

\begin{lstlisting}[style=htmlStyle, caption={base/global/base.html}]
    

        
        <h1>  BASE </h1>
        
        
        
        

        <main class="content">
            
              
        </main>
        
        
        
    </body>
</html>
\end{lstlisting}



\end{document}
\section{}

\begin{lstlisting}[style=pythonStyle, caption={Exemplo em Python}]
\end{lstlisting}

\begin{lstlisting}[style=cssStyle, caption={global/css/red.css}]
\end{lstlisting}

\begin{lstlisting}[style=htmlStyle, caption={base/global/partials/head.html}]
   
\end{lstlisting}

% source amb/Scripts/activate; cd aulas/ola_django
% python manage.py runserver





