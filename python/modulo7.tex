
    \documentclass{article}
    \usepackage[legalpaper, left=1 cm, right=1cm, top=0.5cm, bottom=0.5cm] {geometry}
    \date{} % Remove a exibição da data
    \usepackage{xcolor}
    \usepackage{listings}
    \usepackage{graphicx}
    \usepackage{hyperref} % Para criar links
    \usepackage[utf8]{inputenc}
    \usepackage[T1]{fontenc}

    \lstset{
    language=python,
    basicstyle=\ttfamily,
    keywordstyle=\bfseries\color{blue},
    commentstyle=\color{blue},
    stringstyle=\color{red!70!black},
    numberstyle=\tiny,
    stepnumber=1,
    numbersep=5pt,
    backgroundcolor=\color{white},
    breaklines=true,
    breakautoindent=true,
    showspaces=false,
    showstringspaces=false,
    showtabs=false,
    tabsize=2,
    literate={~}{{\textasciitilde}}1, % Trata ~ como um caractere normal
    extendedchars=true, % Permite caracteres estendidos (acentos, etc.)
    inputencoding=utf8, % Define a codificação de entrada como UTF-8
    literate={á}{{\'a}}1 {ã}{{\~a}}1 {ç}{{\c{c}}}1
    }
    \title{Modulo 7}
    \begin{document}
    \maketitle
    \tableofcontents
    \section{introduction}
    \begin{lstlisting}
        django-admin  startproject project .# inicia o projeto django com o manage.py na raiz
        python manage.py runserver #sobe o servidor
    \end{lstlisting}
    \textbf{http codes}
    \textcolor{red}{https://developer.mozilla.org/pt-BR/docs/Web/HTTP/Status#respostas_de_erro_do_cliente}
    


    

    \end{document}
    \section{}
    \begin{lstlisting}
    \end{lstlisting}

